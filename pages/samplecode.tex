\section{Sample Code}

\subsection{Abstract Factory}
\lstinputlisting[language=C++, firstline=79, lastline=116]{code/Factories.hpp}\label{code:abstract-factory}

\subsection{Adapter}
\lstinputlisting[language=C++]{code/Adapter.hpp}\label{code:adapter}

\subsection{Builder}
\lstinputlisting[language=C++, firstline=5]{code/Builder.hpp}\label{code:builder}

\subsection{Composition Over Inheritance}
\subsubsection{Components}
\lstinputlisting[language=C++, lastline=26]{code/CompositionOverInheritance_1.hpp}\label{code:coi-component}
\subsubsection{Usage}
\lstinputlisting[language=C++, firstline=28]{code/CompositionOverInheritance_1.hpp}\label{code:coi-usage}

\subsection{Decorator}\label{code:decorator}
\subsubsection{Decorator Interface}
The interface class establishes that both core and decorator classes would be a widget, and would be drawn to the screen.
\lstinputlisting[language=C++, firstline=4, lastline=8]{code/Decorator.hpp}
\subsubsection{Core}
The core class, without any added ``decorations'' should hold a width, height, and some text information to display.
\lstinputlisting[language=C++, firstline=10, lastline=31]{code/Decorator.hpp}
\subsubsection{Decorator}
The decorator classes holds a member with a type of widget interface such that it can delegate the \texttt{draw()} function call to the other decorator or core classes. In addition, the derived class also perform extra functionalities that is unique to that decorator class.
\lstinputlisting[language=C++, firstline=33, lastline=65]{code/Decorator.hpp}
\subsubsection{Usage}
\lstinputlisting[language=C++, firstline=67, lastline=82]{code/Decorator.hpp}

\subsection{Observer}\label{code:observer}
\subsubsection{Subject Abstract Observer Class}
The simple implementation of a observer pattern is as follows. Notice that the subject is loosely coupled to the observer. But the observer is tightly coupled to the subject.
\lstinputlisting[language=C++, lastline=72]{code/Observer.hpp}
\subsubsection{Concrete Observers}
Here are some examples of concrete observers that inherits the observer base class. They override the virtual \texttt{handleEvent} method to handle whatever they're designed to do. In this case, any instances of \textit{ZeroObservers} registered to a subject would print out ``Subject received data=0'' whenever the something calls subject's \texttt{setData()} method and passes in 0.
\lstinputlisting[language=C++, firstline=74, lastline=94]{code/Observer.hpp}
For more flexible applications, the observer also could also get a subject pointer and pull the data from the subject upon an event. Consider the following concrete observer class:
\lstinputlisting[language=C++, firstline=96, lastline=108]{code/Observer.hpp}
We invoke \texttt{getSubject()} method (to ensure we can make no changes to the subject itself) and call the getter directly.
\lstinputlisting[language=C++, firstline=111]{code/Observer.hpp}


\subsection{Simple Factory}
\lstinputlisting[language=C++, firstline=3, lastline=59]{code/Factories.hpp}\label{code:simple-factory}
\subsubsection{Not Using Factory}
Not only that we need to create the objects at first, we also need to ensure that every object is properly initialized correctly. In this case, we're using a for-loop. However, one can see that as number of things to initialize increase, and initialization gets more complex, the client would need to do much more work.
\lstinputlisting[language=C++, firstline=62, lastline=71]{code/Factories.hpp}
\subsubsection{Using Factory}
Notice that the initialization logic is hidden away at this level. So using the code is much simpler.
\lstinputlisting[language=C++, firstline=73, lastline=79]{code/Factories.hpp}

\subsection{Singleton}
\lstinputlisting[language=C++]{code/Singleton.hpp}\label{code:singleton}